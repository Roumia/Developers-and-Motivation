\documentclass[11pt]{scrartcl}
\usepackage{atbegshi,picture}
\usepackage{lipsum}

\AtBeginShipout{\AtBeginShipoutUpperLeft{%
  \put(\dimexpr\paperwidth-1cm\relax,-1.5cm){\makebox[3pt][r]{Master on Free Software 2013/2014}}%
}}

\title{Exercises \# 1  The Cathedral and the Bazaar }
\begin{document}
\maketitle
Revise the content of Eric S. Raymond's essay to answer the following points:
	\begin{enumerate}
		\item In your opinion, what are the two most important lessons from the essay?\\
		
			\textbf{Answer:} In my opinion I  think lessons (1,8)\\
			
\textbf{ 1. Every good work of software starts by scratching a developer's personal itch.}\\
There is an arabic wisdom says \emph{"nothing itch your skin like your fingernail"}
I think this is right in everything in   not just programs,  because you  only know what is your exact needs.this will be motivation for starting a free software project.\\
\textbf{ 8. Given a large enough beta-tester and co-developer base, almost every problem will be characterised quickly and the fix obvious to someone.}\\
There is an Arabic wisdom says\emph{"one hand can't claps"} \\ I think this rule is very important for a developers in general they need more tests before release , and more help to show the weakness of the codes .\\
		\item Consider the Android project. Is it a cathedral or a bazaar? Justify your answer appropriately.\\
	It is not bazaar neither  cathedral ,I think Google believe in the “\textbf{cathedral}” development model .
	in other hand in Android we can see  some of  the \textbf{Bazaar model} which is 
\begin{itemize}
\item No clear leader.
\item Many developers with access to repository.
\item Open environment.
\item Frequent releases.
\end{itemize}
	\end{enumerate}
\end{document}