\documentclass[13pt]{scrartcl}
\title{Exercises \# 2  The Reader to Leader Framework }
\author{Amal Roumi}
\begin{document}
\maketitle
Read the paper "The Reader-to-Leader Framework: Motivating Technology-Mediated Social  Participation". Provide 2 additional concrete examples of participation in open online communities that match the different stages suggested in this framework.
This will serve as a starting model to study different motivations of participants in FLOSS development communities.\\
 \textbf{Answer} \\
The successive levels of social participation that we are concerned with can  be categorized as:
\begin{itemize}
  \item \textbf{Reader}: Venturing in, reading, browsing, searching, returning

  \item\textbf{Contributor}: Rating, tagging, reviewing, posting, uploading

  \item\textbf{Collaborator}: Developing relationships, working together, setting goals

  \item\textbf{Leader} : Promoting participation, mentoring novices, setting and upholding
policies
\end{itemize}
	\begin{enumerate}
		\item \textbf{Android Enthusiasts Stack Exchange}\\
Is a question and answer site for enthusiasts and power users of the Android operating system,it's not a discussion forum.
	\item \textbf {TeX - LaTeX} is a question and answer site for users of TeX, LaTeX, ConTeXt, and related typesetting systems.\\
	They have the same structures :
			\begin{itemize}
		\item {\textbf{Reader}}: Any one who is looking for  or searching for answers for his problems 
		\item {\textbf{Contributor}}: Any one who post a question or an answer or vote for the answers.
		\item {\textbf{Collaborator}}:Any one has the ability to vote, comment, and even edit other people's posts "\emph{Privileges control}"
		\item {\textbf{Leader}}: At the highest levels  \emph{when others vote up} , they'll have access to special moderation tools. they'll be able to work alongside the community leaders .
	\end{itemize}
	

	\end{enumerate}
\end{document}